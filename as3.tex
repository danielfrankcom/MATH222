\documentclass{article}
\usepackage{amsmath}
\usepackage{amsfonts}
\usepackage{amssymb}
\usepackage{enumitem}
\usepackage{tikz}
\usepackage{mathtools}

\title{MATH 222 - Assignment 3}
\date{March 2017}
\author{Daniel Frankcom}

\begin{document}
	\pagenumbering{gobble}
	\maketitle
	\setlength{\parindent}{0pt}
	\newcommand{\forceindent}{\leavevmode{\parindent=72pt\indent}}
	\newpage
	\pagenumbering{arabic}
	
	\begin{enumerate}
		\item 
		\begin{enumerate}
			\item This problem can be modeled as the number of ways to put a plus sign between 16 1s, which is equivalent to $2^{n-1}=2^{16-1}=2^{15}=32768$ compositions.
			
			\item This problem can be modeled in the same manner, however this time we will treat each set of 2 1s (1+1) as an object, since our summands must be even.
			\newline $\therefore 2^{(\frac{n}{2})-1}=2^{8-1}=2^7=128$ compositions 
		\end{enumerate}
		
		\item
		\begin{enumerate}
			\item Since we are limited to exactly 7 runs, we cannot have 2 runs of A next to each other, as this would result in less than exactly 7. Therefore there are 2 cases:
			\begin{enumerate}
				\item There are 3 runs of A and 4 runs of B
				\newline In this case, we must distribute 7 wins for A among 3 runs, which is equivalent to counting the number of integer solutions to $x_1+x_2+x_3=7$, where $x_i$ represents the run of A in the list from left to right.
				\newline We must ensure however that each bin has at least one element, so we can modify our question slightly to be number of integer solutions to $y_1+y_2+y_3=4$, assuming that each run already contains 1 element.
				\newline $\therefore {{w+r-1}\choose{r-1}}={{4+3-1}\choose{3-1}}={6\choose2}$
				\newline We must also distribute 8 wins for B among 4 runs, which using the same process as above, yields ${{5+4-1}\choose{4-1}}={8\choose3}$
				\newline We can then multiply these together, since for arrangement of runs of A, there are ${8\choose3}$ runs of B.
				\newline $\therefore$ the number of possible winner lists when there are 3 runs of A and 4 runs of B is ${6\choose2}{8\choose3}=(15)(56)=840$
				
				\item There are 4 runs of A and 3 runs of B
				\newline Using a similar process to that above, we get ${7\choose3}{7\choose2}=735$
			\end{enumerate}
			We can then add these 2 possibilities together to compute the total number of list outcomes: $840+735=1575$ lists
			
			\item Here we must distribute 15 wins among 7 runs, which is equivalent to counting the number of integer solutions to $x_1+x_2+\dots+x_7=15$, where $x_i$ represents the run number from left to right.
			\newline We know that there must be at least 1 win each each run however, as otherwise there would be less than the required 7 runs.
			\newline $\therefore$ we can rewrite our equation to be $x_1+x_2+\dots+x_7=8$
			\newline The number of integers that satisfy this is ${{w+r-1}\choose{r-1}}={{8+7-1}\choose{7-1}}={14\choose6}$
			\newline This means that we have ${14\choose6}$ arrangements of wins for our runs, however we know that there cannot be an A run and a B run next to each other, as this would diminish the total number of runs.
			\newline Therefore our runs must alternate A and B, so there are 2 cases:
			\begin{enumerate}
				\item Our list starts with an A run
				\item Our list starts with a B run
			\end{enumerate}
			For each of these cases there are ${14\choose6}$ ways to arrange our wins, so there are $2{14\choose6}$ possible lists in total.
			\newline $\therefore 2{14\choose6}=2(3003)=6006$ possible lists
		\end{enumerate}
	
		\item Since we have lower bounds on $x_3$ and $x_4$, we can rewrite the equation by assuming that $x_3$ already contains 3, and that $x_4$ already contains 4.
		\newline The equation then becomes $x_1+x_2+x_3+x_4=15$
		\newline Due to the upper bounds on our integers, this problem can now be written as an inclusion/exclusion question.
		\newline
		\newline Let $N$ be the total number of integer solutions to $x_1+x_2+x_3+x_4=15$ with no restrictions on any of the variables.
		\newline $\therefore {{r+n-1}\choose{n-1}}={{15+4-1}\choose{4-1}}={18\choose3}=816$
		\newline Let $C_1$ be the condition that $x_1\geq6$
		\newline Let $C_2$ be the condition that $x_2\geq7$
		\newline Let $C_3$ be the condition that $x_3\geq7$
		\newline Let $C_4$ be the condition that $x_4\geq6$
		\newline Then the answer to this question can be computed by $N(\bar{C_1}\bar{C_2}\bar{C_3}\bar{C_4})$
		\newline
		\newline $N(C_1)$ is the number of solutions to $x_1+x_2+x_3+x_4=15$ where $x_1\geq6$
		\newline This can be re-written as $x_1+x_2+x_3+x_4=9$ where $x_1$ has no restrictions
		\newline $\therefore {{r+n-1}\choose{n-1}}={{9+4-1}\choose{4-1}}={12\choose3}=220$
		\newline $N(C_2)$ is the solutions to $x_1+x_2+x_3+x_4=8$ where $x_2$ has no restrictions
		\newline $\therefore {{r+n-1}\choose{n-1}}={{8+4-1}\choose{4-1}}={11\choose3}=165$
		\newline $N(C_3)$ and $N(C_4)$ have the same number of solutions as $N(C_2)$ and $N(C_1)$
		\newline
		\newline $N(C_iC_j)$ is shown by $x_1+x_2+x_3+x_4=15$ where $x_i\geq6$ and $x_j\ge7$
		\newline This is equivalent to $x_1+x_2+x_3+x_4=2$ with no restrictions
		\newline $\therefore {{r+n-1}\choose{n-1}}={{2+4-1}\choose{4-1}}={5\choose3}=10$
		\newline There are 4 such combinations of this case set.
		\newline $N(C_iC_j)$ is shown by $x_1+x_2+x_3+x_4=15$ where $x_i,x_j\geq6$
		\newline This is equivalent to $x_1+x_2+x_3+x_4=3$ with no restrictions
		\newline $\therefore {{r+n-1}\choose{n-1}}={{3+4-1}\choose{4-1}}={6\choose3}=20$
		\newline There are 2 such combinations of this case set.
		\newline $N(C_iC_j)$ is shown by $x_1+x_2+x_3+x_4=15$ where $x_i,x_j\geq7$
		\newline This is equivalent to $x_1+x_2+x_3+x_4=1$ with no restrictions
		\newline $\therefore {{r+n-1}\choose{n-1}}={{1+4-1}\choose{4-1}}={4\choose3}=4$
		\newline There are 2 such combinations of this case set.
		\newline
		\newline The intersections containing both 3 and 4 conditions all have 0 solutions, as there is no way to sum 3 values  that follow any of the conditions and achieve a result that is equal to 15.
		\newline
		\newline $\bar{N}=N-\sum N(C_i)+\sum N(C_iC_j)$
		\newline $=816-(2*220+2*165)+(4*10+2*20+2*4)=816-770$
		\newline $=88$ integer solutions within the given bounds
		
		\newpage
		\item Let $N$ be the number of permutations of the 9 digits, or 9!
		\newline Let $C_i$ be the condition that digit $i$ is immediately followed by $i+1$, where $i=1,2,\dots,8$
		\newline Then the answer to this question can be computed by $N(\bar{C_1}\bar{C_2}\dots\bar{C_8})$
		\newline
		\newline $N(C_i)$ can be modeled by grouping together $i$ and $i+1$, treating them as a single digit. By doing this, the number of permutations becomes 8!
		\newline $N(C_iC_j)$ can be modeled by grouping together the 2 sets of digits, resulting in 7! Note that it does not matter if we have 3 digits consecutively, as the resulting number of movable objects remains the same.
		\begin{center}
			$\vdots$
		\end{center}
		$N(C_1C_2\dots C_8)$ results in 1! as by this point all digits are grouped.
		\newline
		\newline $\bar{N}=N-\sum N(C_i)+\sum N(C_iC_j)-\dots+N(C_iC_j\dots C_8)$
		\newline $=9!-(8)8!+(7)7!-\dots+1$
		\newline $=362880-322560+35280-4320+600-96+18-4+1$
		\newline $=71799$ permutations
		
		\item $\sum\limits_{k=0}^{n-1}(-1)^k{n\choose k}(n-k)^m$ can be used to model the number of onto functions 2 given sets, and in our case $n=m=9$
		\newline $=\sum\limits_{k=0}^{8}(-1)^k{8\choose k}(8-k)^9$
		\newline $=8^9-(8)7^9+(28)6^9-(56)5^9+(70)4^9-(56)3^9+(28)2^9-8$
		\newline $=1451520$ functions
		\newline We then need to remove the functions in which $f(x)=x$ for all even x.
		\newline There are 4 even numbers in the set, and each represents the number of onto mappings from a set of size 8 to another set of size 8.
		\newline Therefore the number of onto mappings in which $f(x)=x$ for any x is:
		\newline $=4\sum\limits_{k=0}^{7}(-1)^k{7\choose k}(7-k)^8$
		\newline $=4(7^8-(7)6^8+(21)5^8-(35)4^8+(35)3^8-(21)2^8+7)$
		\newline $=4(141120)$
		\newline $=564480$ functions
		\newline $\therefore 1451520-564480$
		\newline $=887040$ functions that fulfill the condition
		
		\item When dividing by 12, there are 12 possible remainders, therefore in 13 chosen integers, at least 2 must have the same remainder.
		\newline Our integers $a,b$ can be written as $a=12k_1 + r$ and $b=12k_2+r$
		\newline $\therefore a-b=12k_1+r-12k_2-r$
		\newline $\therefore a-b=12(k_1-k_2)$
		\newline $\therefore 12(a-b)=k_1-k_2$ which implies that $12|a-b$ since $k_1-k_2$ is an integer
		
	
	\end{enumerate}
\end{document}
\documentclass{article}
\usepackage{amsmath}
\usepackage{amsfonts}
\usepackage{amssymb}
\usepackage{enumitem}
\usepackage{tikz}
\usepackage{mathtools}

\title{MATH 222 - Assignment 4}
\date{March 2017}
\author{Daniel Frankcom}

\begin{document}
	\pagenumbering{gobble}
	\maketitle
	\setlength{\parindent}{0pt}
	\newcommand{\forceindent}{\leavevmode{\parindent=72pt\indent}}
	\newpage
	\pagenumbering{arabic}
	
	\begin{enumerate}
		
		\item 
		\begin{enumerate}
			\item In order for a relation to be both symmetric and antisymmetric, it must only contain reflexive elements. For each reflexive pair $(x,x)$, there are 2 options: either the pair is in the relation, or it is not.
			\newline $\therefore$ there are $2^n$ relations on $A$ that fulfill the requirements
			\item If $R$ is a relation on $A$ that is antisymmetric, then it may not contain any 2 pairs such that $(x,y)$ and $(y,x)$ are both in $R$ unless $x=y$.
			\newline $\therefore$ we can count the number of elements in the upper diagonal half of the corresponding relation matrix.
			\newline This is equivalent to $\sum\limits_{i=1}^{n}i=\frac{n(n+1)}{2}$
		\end{enumerate}
	
	\end{enumerate}
\end{document}